\documentclass[]{article}
\usepackage{lmodern}
\usepackage{amssymb,amsmath}
\usepackage{ifxetex,ifluatex}
\usepackage{fixltx2e} % provides \textsubscript
\ifnum 0\ifxetex 1\fi\ifluatex 1\fi=0 % if pdftex
  \usepackage[T1]{fontenc}
  \usepackage[utf8]{inputenc}
\else % if luatex or xelatex
  \ifxetex
    \usepackage{mathspec}
  \else
    \usepackage{fontspec}
  \fi
  \defaultfontfeatures{Ligatures=TeX,Scale=MatchLowercase}
\fi
% use upquote if available, for straight quotes in verbatim environments
\IfFileExists{upquote.sty}{\usepackage{upquote}}{}
% use microtype if available
\IfFileExists{microtype.sty}{%
\usepackage{microtype}
\UseMicrotypeSet[protrusion]{basicmath} % disable protrusion for tt fonts
}{}
\usepackage[margin=1in]{geometry}
\usepackage{hyperref}
\hypersetup{unicode=true,
            pdftitle={Lionfish Cluster Plot},
            pdfauthor={GG},
            pdfborder={0 0 0},
            breaklinks=true}
\urlstyle{same}  % don't use monospace font for urls
\usepackage{color}
\usepackage{fancyvrb}
\newcommand{\VerbBar}{|}
\newcommand{\VERB}{\Verb[commandchars=\\\{\}]}
\DefineVerbatimEnvironment{Highlighting}{Verbatim}{commandchars=\\\{\}}
% Add ',fontsize=\small' for more characters per line
\usepackage{framed}
\definecolor{shadecolor}{RGB}{248,248,248}
\newenvironment{Shaded}{\begin{snugshade}}{\end{snugshade}}
\newcommand{\KeywordTok}[1]{\textcolor[rgb]{0.13,0.29,0.53}{\textbf{#1}}}
\newcommand{\DataTypeTok}[1]{\textcolor[rgb]{0.13,0.29,0.53}{#1}}
\newcommand{\DecValTok}[1]{\textcolor[rgb]{0.00,0.00,0.81}{#1}}
\newcommand{\BaseNTok}[1]{\textcolor[rgb]{0.00,0.00,0.81}{#1}}
\newcommand{\FloatTok}[1]{\textcolor[rgb]{0.00,0.00,0.81}{#1}}
\newcommand{\ConstantTok}[1]{\textcolor[rgb]{0.00,0.00,0.00}{#1}}
\newcommand{\CharTok}[1]{\textcolor[rgb]{0.31,0.60,0.02}{#1}}
\newcommand{\SpecialCharTok}[1]{\textcolor[rgb]{0.00,0.00,0.00}{#1}}
\newcommand{\StringTok}[1]{\textcolor[rgb]{0.31,0.60,0.02}{#1}}
\newcommand{\VerbatimStringTok}[1]{\textcolor[rgb]{0.31,0.60,0.02}{#1}}
\newcommand{\SpecialStringTok}[1]{\textcolor[rgb]{0.31,0.60,0.02}{#1}}
\newcommand{\ImportTok}[1]{#1}
\newcommand{\CommentTok}[1]{\textcolor[rgb]{0.56,0.35,0.01}{\textit{#1}}}
\newcommand{\DocumentationTok}[1]{\textcolor[rgb]{0.56,0.35,0.01}{\textbf{\textit{#1}}}}
\newcommand{\AnnotationTok}[1]{\textcolor[rgb]{0.56,0.35,0.01}{\textbf{\textit{#1}}}}
\newcommand{\CommentVarTok}[1]{\textcolor[rgb]{0.56,0.35,0.01}{\textbf{\textit{#1}}}}
\newcommand{\OtherTok}[1]{\textcolor[rgb]{0.56,0.35,0.01}{#1}}
\newcommand{\FunctionTok}[1]{\textcolor[rgb]{0.00,0.00,0.00}{#1}}
\newcommand{\VariableTok}[1]{\textcolor[rgb]{0.00,0.00,0.00}{#1}}
\newcommand{\ControlFlowTok}[1]{\textcolor[rgb]{0.13,0.29,0.53}{\textbf{#1}}}
\newcommand{\OperatorTok}[1]{\textcolor[rgb]{0.81,0.36,0.00}{\textbf{#1}}}
\newcommand{\BuiltInTok}[1]{#1}
\newcommand{\ExtensionTok}[1]{#1}
\newcommand{\PreprocessorTok}[1]{\textcolor[rgb]{0.56,0.35,0.01}{\textit{#1}}}
\newcommand{\AttributeTok}[1]{\textcolor[rgb]{0.77,0.63,0.00}{#1}}
\newcommand{\RegionMarkerTok}[1]{#1}
\newcommand{\InformationTok}[1]{\textcolor[rgb]{0.56,0.35,0.01}{\textbf{\textit{#1}}}}
\newcommand{\WarningTok}[1]{\textcolor[rgb]{0.56,0.35,0.01}{\textbf{\textit{#1}}}}
\newcommand{\AlertTok}[1]{\textcolor[rgb]{0.94,0.16,0.16}{#1}}
\newcommand{\ErrorTok}[1]{\textcolor[rgb]{0.64,0.00,0.00}{\textbf{#1}}}
\newcommand{\NormalTok}[1]{#1}
\usepackage{graphicx,grffile}
\makeatletter
\def\maxwidth{\ifdim\Gin@nat@width>\linewidth\linewidth\else\Gin@nat@width\fi}
\def\maxheight{\ifdim\Gin@nat@height>\textheight\textheight\else\Gin@nat@height\fi}
\makeatother
% Scale images if necessary, so that they will not overflow the page
% margins by default, and it is still possible to overwrite the defaults
% using explicit options in \includegraphics[width, height, ...]{}
\setkeys{Gin}{width=\maxwidth,height=\maxheight,keepaspectratio}
\IfFileExists{parskip.sty}{%
\usepackage{parskip}
}{% else
\setlength{\parindent}{0pt}
\setlength{\parskip}{6pt plus 2pt minus 1pt}
}
\setlength{\emergencystretch}{3em}  % prevent overfull lines
\providecommand{\tightlist}{%
  \setlength{\itemsep}{0pt}\setlength{\parskip}{0pt}}
\setcounter{secnumdepth}{0}
% Redefines (sub)paragraphs to behave more like sections
\ifx\paragraph\undefined\else
\let\oldparagraph\paragraph
\renewcommand{\paragraph}[1]{\oldparagraph{#1}\mbox{}}
\fi
\ifx\subparagraph\undefined\else
\let\oldsubparagraph\subparagraph
\renewcommand{\subparagraph}[1]{\oldsubparagraph{#1}\mbox{}}
\fi

%%% Use protect on footnotes to avoid problems with footnotes in titles
\let\rmarkdownfootnote\footnote%
\def\footnote{\protect\rmarkdownfootnote}

%%% Change title format to be more compact
\usepackage{titling}

% Create subtitle command for use in maketitle
\newcommand{\subtitle}[1]{
  \posttitle{
    \begin{center}\large#1\end{center}
    }
}

\setlength{\droptitle}{-2em}
  \title{Lionfish Cluster Plot}
  \pretitle{\vspace{\droptitle}\centering\huge}
  \posttitle{\par}
  \author{GG}
  \preauthor{\centering\large\emph}
  \postauthor{\par}
  \predate{\centering\large\emph}
  \postdate{\par}
  \date{1/28/2019}


\begin{document}
\maketitle

\begin{Shaded}
\begin{Highlighting}[]
\KeywordTok{library}\NormalTok{(vegan)}
\end{Highlighting}
\end{Shaded}

\begin{verbatim}
## Warning: package 'vegan' was built under R version 3.4.4
\end{verbatim}

\begin{verbatim}
## Loading required package: permute
\end{verbatim}

\begin{verbatim}
## Loading required package: lattice
\end{verbatim}

\begin{verbatim}
## This is vegan 2.5-1
\end{verbatim}

\begin{Shaded}
\begin{Highlighting}[]
\KeywordTok{library}\NormalTok{(dbplyr)}
\KeywordTok{library}\NormalTok{(tidyverse)}
\end{Highlighting}
\end{Shaded}

\begin{verbatim}
## -- Attaching packages ------------------------------------------------------------------------------------------ tidyverse 1.2.1 --
\end{verbatim}

\begin{verbatim}
## √ ggplot2 2.2.1     √ purrr   0.2.4
## √ tibble  1.4.2     √ dplyr   0.7.4
## √ tidyr   0.8.0     √ stringr 1.3.0
## √ readr   1.1.1     √ forcats 0.3.0
\end{verbatim}

\begin{verbatim}
## -- Conflicts --------------------------------------------------------------------------------------------- tidyverse_conflicts() --
## x dplyr::filter() masks stats::filter()
## x dplyr::ident()  masks dbplyr::ident()
## x dplyr::lag()    masks stats::lag()
## x dplyr::sql()    masks dbplyr::sql()
\end{verbatim}

\begin{Shaded}
\begin{Highlighting}[]
\NormalTok{PCA_fish <-}\StringTok{ }\KeywordTok{read_csv}\NormalTok{(}\StringTok{"~/Desktop/Lionfish/Darwin_matrix.csv"}\NormalTok{)}
\end{Highlighting}
\end{Shaded}

\begin{verbatim}
## Warning: Missing column names filled in: 'X1' [1]
\end{verbatim}

\begin{verbatim}
## Parsed with column specification:
## cols(
##   .default = col_double(),
##   X1 = col_character()
## )
\end{verbatim}

\begin{verbatim}
## See spec(...) for full column specifications.
\end{verbatim}

\begin{Shaded}
\begin{Highlighting}[]
\KeywordTok{library}\NormalTok{(tidyverse)}
\KeywordTok{row.names}\NormalTok{(PCA_fish) <-}\StringTok{ }\NormalTok{PCA_fish}\OperatorTok{$}\NormalTok{X1}
\end{Highlighting}
\end{Shaded}

\begin{verbatim}
## Warning: Setting row names on a tibble is deprecated.
\end{verbatim}

\begin{Shaded}
\begin{Highlighting}[]
\NormalTok{PCA_fish[}\DecValTok{1}\NormalTok{] =}\StringTok{ }\OtherTok{NULL}
\end{Highlighting}
\end{Shaded}

K means cluster analysis

\begin{Shaded}
\begin{Highlighting}[]
\KeywordTok{library}\NormalTok{(tidyverse)  }\CommentTok{# data manipulation}
\KeywordTok{library}\NormalTok{(cluster)    }\CommentTok{# clustering algorithms}
\KeywordTok{library}\NormalTok{(factoextra)}
\end{Highlighting}
\end{Shaded}

\begin{verbatim}
## Welcome! Related Books: `Practical Guide To Cluster Analysis in R` at https://goo.gl/13EFCZ
\end{verbatim}

\begin{Shaded}
\begin{Highlighting}[]
\KeywordTok{library}\NormalTok{(ggplot2)}
\NormalTok{cluster2 <-}\StringTok{ }\KeywordTok{kmeans}\NormalTok{(PCA_fish, }\DataTypeTok{centers =} \DecValTok{2}\NormalTok{, }\DataTypeTok{nstart =} \DecValTok{25}\NormalTok{)}
\KeywordTok{str}\NormalTok{(cluster2)}
\end{Highlighting}
\end{Shaded}

\begin{verbatim}
## List of 9
##  $ cluster     : Named int [1:11] 1 1 1 1 1 2 2 2 1 2 ...
##   ..- attr(*, "names")= chr [1:11] "Elbow" "Tiger" "XL" "Spittal" ...
##  $ centers     : num [1:2, 1:29] 0.0773 0.4283 0.3028 0.0669 0.1872 ...
##   ..- attr(*, "dimnames")=List of 2
##   .. ..$ : chr [1:2] "1" "2"
##   .. ..$ : chr [1:29] "Chromis bermudae" "Chromis insolata" "Paranthia furcifer" "Chaetodon sedentarius" ...
##  $ totss       : num 1.56
##  $ withinss    : num [1:2] 0.851 0.171
##  $ tot.withinss: num 1.02
##  $ betweenss   : num 0.539
##  $ size        : int [1:2] 7 4
##  $ iter        : int 1
##  $ ifault      : int 0
##  - attr(*, "class")= chr "kmeans"
\end{verbatim}

\begin{Shaded}
\begin{Highlighting}[]
\NormalTok{pcaplot <-}\StringTok{ }\KeywordTok{fviz_cluster}\NormalTok{(cluster2, }\DataTypeTok{data =}\NormalTok{ PCA_fish, }\DataTypeTok{fill =} \StringTok{"darkblue"}\NormalTok{) }\OperatorTok{+}\StringTok{ }\KeywordTok{ylim}\NormalTok{(}\OperatorTok{-}\FloatTok{2.5}\NormalTok{,}\FloatTok{4.5}\NormalTok{) }\OperatorTok{+}\StringTok{ }\KeywordTok{theme_classic}\NormalTok{()}
\KeywordTok{ggsave}\NormalTok{(PCA_fish, }\DataTypeTok{plot =}\NormalTok{ pcaplot, }\DataTypeTok{device =} \StringTok{"eps"}\NormalTok{)}
\end{Highlighting}
\end{Shaded}

\begin{verbatim}
## Saving 6.5 x 4.5 in image
\end{verbatim}

\begin{verbatim}
## Warning in grid.Call.graphics(C_polygon, x$x, x$y, index): semi-
## transparency is not supported on this device: reported only once per page
\end{verbatim}

\begin{Shaded}
\begin{Highlighting}[]
\NormalTok{pcaplot}
\end{Highlighting}
\end{Shaded}

\includegraphics{Lionfish_final_cluster_analysis_files/figure-latex/unnamed-chunk-5-1.pdf}

\begin{Shaded}
\begin{Highlighting}[]
\NormalTok{ggpcaplot <-}\StringTok{ }\NormalTok{pcaplot }\OperatorTok{+}\StringTok{ }\KeywordTok{theme_classic}\NormalTok{() }\OperatorTok{+}\StringTok{ }\KeywordTok{labs}\NormalTok{(}\DataTypeTok{title=}\StringTok{""}\NormalTok{)}
\NormalTok{ggpcaplot}
\end{Highlighting}
\end{Shaded}

\includegraphics{Lionfish_final_cluster_analysis_files/figure-latex/unnamed-chunk-6-1.pdf}

\begin{Shaded}
\begin{Highlighting}[]
\KeywordTok{ggsave}\NormalTok{(PCA_fish, }\DataTypeTok{plot =}\NormalTok{ ggpcaplot, }\DataTypeTok{device =} \StringTok{"eps"}\NormalTok{)}
\end{Highlighting}
\end{Shaded}

\begin{verbatim}
## Saving 6.5 x 4.5 in image
\end{verbatim}

\begin{verbatim}
## Warning in grid.Call.graphics(C_polygon, x$x, x$y, index): semi-
## transparency is not supported on this device: reported only once per page
\end{verbatim}


\end{document}
